\documentclass[13p, a4paper, oneside]{article}

\usepackage[utf8]{inputenc}
\usepackage[MeX]{polski}
\usepackage{graphicx}
\usepackage{csvsimple}
\usepackage[section]{placeins}
\usepackage[top=0.5in, bottom=0.5in, left=1in, right=1in]{geometry}
\usepackage{textcomp}

\begin{document}
\title{The Cake router}
\author{Joanna Rymko \and Krzysztof Marut}
\maketitle
\paragraph{} Projekt polega na pozwoleniu na wysyłanie plików za pomocą pośredników, używając protokołu UDP. Ma on spełniać wymagania:
\begin{itemize}
\item Umożliwiamy przesyłanie komunikatów UDP
\item Klient posiada listę wszystkich węzłów pośredniczących
\item Wybiera w niej trasę przez N węzłów pośredniczących
\item Do P1 wysyła komunikat zawierający listę węzłów pośrednich, ostatecznego odbiorce oraz dane do przesłania.
\item Każdy z węzłów PN pamięta od kogo dostał w danym połączeniu informację.
\item Możliwe by klient wysłał pakiet UDP od siebie do innego hosta (i otrzymał odpowiedź). Pakiet (i odpowiedź) musi przejść przez dwa węzły pośredniczące oraz węzeł końcowy.
\item Sieć jest w stanie obsłużyć dwóch równolegle pracujących klientów.
\end{itemize}
\section{Specyfikacja}
\paragraph{} Program klienta na początku działania wysyłać będzie (broadcast) informację na port, na którym nasłuchiwać będą wszystkie uruchomione węzły. Te w odpowiedzi wyślą informację zwrotną, która posłuży jako baza węzłów, spośród których będzie można wybrać dowolną ilość\footnote{Rozważane jest stworzenie rudymentarnego GUI by ułatwić korzystanie z programu}.
\paragraph{}Wybrane węzły zostaną złożone najprawdopodniej za pomocą LinkedList. Zawierały będą one informację o adresie kolejnego węzła. Klient, i każdy kolejny węzeł zdejmie z kolejki element, i użyje przechowywanego tam adresu, by wysłać do niego resztę listy. Każdy z węzłów bedzie też przechowywał adres, z którego lista została do niego przesłana. Ostanim elementem listy będzie odbiorca wiadomości. Oczywiście poza listą przesyłany też będzie plik, przekonwertowany uprzednio na strumień bajtów.
\subparagraph{}Jeśli klient końcowy otrzyma plik, wyśle informacje zwrotną, która przejdzie tą samą trasę w odwrotnym kierunku. 
\paragraph{}By zapewnić obsługę więcej niż jednego klienta, we wszystkich warstwach wykorzystywane będą wątki runnable.
\paragraph{}Odbiorca wiadomości, po jej otrzymaniu, poproszony zostanie o podanie ścieżki, do której chciałby, aby plik został zapisany. By wyeliminować zapisanie pliku z niewłaściwym rozszerzeniem, odbiorca będzie poinformowany o oryginalnej nazwie i rozszerzeniu pliku (i będzie miał opcję, by tejże nie zmieniać).

\end{document}